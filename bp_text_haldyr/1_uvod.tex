\chapter{Úvod} \label{uvod}

% Úvod charakterizující kontext zadání, případně motivace.
% ----------
% Navrhněte a~implementujte aplikaci, která umožní vytvoření virtuální počítačové sítě, pro potřeby předmětu Y36PSI. Na
% systém se bude možno připojit s~pomocí telnetu. Z~pohledu uživatele se bude systém tvářit jako reálná síť. Zaměřte se
% především na konfiguraci systému Cisco. Systém bude podporovat příkazy potřebné ke konfiguraci síťových rozhraní,
% směrování a~překladu adres. Pro ověření správnosti konfigurace implementujte příkaz ping a~traceroute.

Úkolem této práce je navrhnout a implementovat aplikaci, která umožní vytvoření virtuální počítačové sítě pro předmět Y36PSI\footnote{Počítačové sítě}. Z pohledu uživatele se systém musí tvářit jako reálná síť. Tento úkol byl rozdělen na dvě části: Cisco a Linux. Můj úkol je právě emulace Cisco IOS\footnote{Internetwork Operating System je operační systém používaný na směrovačích a přepínačích firmy Cisco Systems}. Na dnešním virtuálním trhu existuje celá řada programů pro virtualizaci sítě. Většina z nich je však špatně dostupných (zejména kvůli licenci) nebo se nehodí pro potřeby předmětu Počítačové sítě. 

Vize je taková, že student si v teple domova spustí tuto aplikaci a \uv{pohraje} si s virtuálním ciscem, ke kterému běžný smrtelník nemá přístup. Zjistí, jak to funguje a pak už jen přijde na cvičení předmětu a vše nakonfiguruje tak, jak to má být. 

Jelikož tento projekt přesahuje rozsah jedné bakalářské práce, tak byly vymezeny hranice, aby se tento úkol mohl rozdělit na dvě části. Nakonec celá aplikace byla rozdělena na části tři. První je část společná, kde je implementováno jádro klient - server. Druhá část je Cisco IOS, tu jsem dostal na starost já\footnote{Oba jsme chtěli programovat linuxovou část, protože s OS Linux máme oba zkušenosti. Po losování \uv{Černý Petr - Cisco} padlo na mne.}. A třetí část je platforma Linux, kterou zpracoval Tomáš Pitřinec v bakalářské práci Simulátor virtuální počítačové sítě Linux.

\section{Struktura práce}
Tady bude popis členění práce na jednotivé sekce.