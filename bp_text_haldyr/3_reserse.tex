\chapter{Existující řešení}
Existujících implementací je na síťovém trhu celá řada, nicméně ne každé řešení je vhodné pro potřeby předmětu Y36PSI.

\section{Cisco Packet Tracer}
Asi nejznámějším simulátorem Cisca je Packet Tracer \cite{cisco:pt} od stejnojmenné firmy. Program slouží k simulaci síťového provozu počítačových sítí založených na hardware od Cisco Systems. Packet Tracer umožňuje vizualizaci provozu sítě, konfiguraci síťových prvků v grafickém i textovém módu. Program obsahuje velmi málo odchylek od skutečného cisco směrovače. Packet Tracer je dostupný pro platformy Windows i Linux zdarma pro všechny členy Cisco Networking Academy. To je zároveň velkou nevýhodou, protože licence neumožňuje jiné použití a je tedy Paket Tracer ve škole nepoužitelný.


\section{OMNeT++ simulátor} 
Simulační systém OMNeT++ \cite{reserse:omnet_hp} je velmi propracovaný opensource nástroj pro simulaci prakticky čehokoliv. OMNeT++ je postaven na modulární architektuře, takže při správných knihovnách (modulech) může simulovat počítačovou síť. Systém dokáže simulovat Cisco IOS i počítač postavený na linuxu. Aplikace je komplikovaná a představu jednoduchého programu pro výuku studentů spíše nesplňuje.

Více se tímto simulátorem zabýval Bc. Jan Michek v rámci své diplomové práce Emulátor počítačové sítě \cite{reserse:omnet_dp}.


\section{Simulation Toolkit 7.0} 
Simulation Toolkit 7.0 \cite{reserse:adventnet} je grafický simulátor pro testování a výuku různých síťových aplikací. Simulation Toolkit umožňuje simulaci více než 50000 SNMP služeb (v1, v2c, v3), TL1, TFTP, FTP, Telnet a Cisco IOS na jediném počítači. Software je nabízen pod shareware licencí a je dostupný pro operační systémy Windows, Linux i Unix. Za poskytnutí osobních údajů lze stáhnout plně funkční zkušební verzi na 30 dní. Plná časově neomezená verze stojí od \$995 do \$14995.


\section{Boson NetSim Network Simulator} 
Boson NetSim Network Simulator \cite{reserse:boson} je aplikace pro simulaci síťového hardware a software a je designován jako výuková pomůcka pro začínající administrátory Cisco IOS. Systém dokáže simulovat více než 40 různých síťových prvků od firmy Cisco Systems. Simulátor obsahuje grafický i textový konfigurační režim sítí. Program je dostupný pro Windows, Linux i Solaris. Podobně jako Simulation Toolkit i tento software je nabízen ve zkušební verzi na 30 dní. Plná časově neomezená verze stojí od \$99 do \$349.


\section{Dynamips Cisco 7200 Simulator}
Dynamips Cisco 7200 Simulator \cite{reserse:dynamips} je program napsaný Christophe Fillot za účelem emulace Cisco směrovačů. Dynamips funguje na platformách Linux, Mac OS X, Windows a emuluje hardware Cisco směrovačů tak, že se načte obraz originálního Cisco IOS do emulátoru. Program tedy umožňuje využívat veškeré funkce Cisco IOS. Program je licencován pod GNU GPL\footnote{licence pro svobodný software}, ale bohužel obraz IOSu není volně k dispozici. Tudíž lze program legálně používat jen jako zákazník firmy Cisco - to ale škola není, takže tento program také není vhodný pro výuku.


\section{Virtuální laboratoř počítačových sítí VirtLab}
Projekt VirtLab \cite{reserse:virtlab} zpřístupňuje laboratorní prvky pro praktickou výuku počítačových sítí vzdáleně prostřednictvím Internetu. Studenti ostravské VŠB-TU\footnote{Vysoká škola báňská - Technická univerzita Ostrava} mají možnost si rezervovat pomocí webového rozhraní laboratorní síťové prvky na určitý časový interval a po té k nim přistupovat přes webový prohlížeč pomocí Java appletů. Propojení síťových prvků se uskuteční automaticky dle zvolené úlohy. Dle mého názoru je to systém velmi ambiciózní, nicméně pro výuku Y36PSI v praxi nepoužitelný\footnote{Za předpokladu že ČVUT nenaváže spolupráci s VŠB-TU Ostrava.}.




