\chapter{Úvod}

% pokyny k mojí práci:
% Navrhněte a implementujte aplikaci, která umožní vytvoření virtuální počítačové sítě, pro potřeby předmětu Y36PSI. Na systém se bude možno připojit s pomocí telnetu. Z pohledu uživatele se bude systém tvářit jako reálná síť. Zaměřte se především na konfiguraci systému Linux. Systém bude podporovat příkazy potřebné ke konfiguraci síťových rozhraní, směrování a překladu adres. Pro ověření správnosti konfigurace implementujte příkaz ping a traceroute.


% původní dlouhej úvod:
% Jednou z laboratorních úloh předmětu Počítačové sítě (Y36PSI) na FELu je postavení sítě mezi několika linuxovými a ciscovými počítači. Studenti, kteří tento úkol vykonávají, nemají často s takovou činností žádnou osobní zkušenost. Z přednášek většinou vědí, jak by měli síť očíslovat, avšak jen několik málo uživatelů linuxu umí nastavit adresy a brány na počítači s linuxem a jen nepatrný počet studentů nastavoval síťové adresy na skutečném Cisco routeru. Studenti pak při laboratořích řeší různé banální problémy, kterým by se mohli vyhnout, kdyby měli možnost zkusit si nastavit podobou síť již před laboratorní úlohou. Kvůli kapacitě laboratoře však není možné, aby si všichni studenti zkoušeli úlohu předem. Proto by se jim mohl hodit simulátor, který by jednoduše spustili na svém počítači a na kterém by si mohli nastavování síťových parametrů na linuxu a ciscu zkusit. Právě návrhem a implementací takového síťového simulátoru počítačů s OS Linux se zabývá tato bakalářská práce. 

% zkrácenej úvod:
Jednou z laboratorních úloh předmětu Počítačové sítě (Y36PSI) na FELu je postavení sítě mezi několika linuxovými a ciscovými počítači. Studenti, kteří tento úkol plní, nemají často s takovou činností žádnou osobní zkušenost, a tak během úlohy řeší různé banální problémy, kterým by se mohli vyhnout, kdyby měli možnost zkusit si nastavit podobou síť již před samotnou laboratorní úlohou. Mohl by se jim hodit simulátor, který by jednoduše spustili na svém počítači a na kterém by si mohli nastavování síťových parametrů na linuxu a ciscu zkusit. Právě návrhem a implementací takového síťového simulátoru počítačů s OS Linux se zabývá tato bakalářská práce. 


\section{Cíle práce}

Cílem práce je v programovacím jazyce Java SE navrhnout a implementovat aplikaci, která umožní vytvoření virtuální počítačové sítě, pro potřeby předmětu Y36PSI. Z pohledu uživatele by aplikace měla vypadat stejně jako reálná síť. Uživatel spustí aplikaci v konsoli a pak se pomocí telnetu připojí k jejím jednotlivým virtuálním počítačům, podobně jako protokolem ssh k počítačům s OS Linux. Aplikace bude podporovat příkazy potřebné ke konfiguraci síťových rozhraní (ifconfig, ip address), směrování (route, ip route) a překladu adres (iptables -t nat). Pro ověření správnosti konfigurace sítě budou implementovány příkazy ping a traceroute.

Nastavenou konfiguraci sítě bude možné uložit do souboru a zase ji ze souboru načíst. Uživatel bude mít možnost vytvářet libovolné sítě s libovolným počtem počítačů typu linux nebo cisco tak, že infrastrukturu sítě napíše do konfiguračního souboru a pak ji z něho načte.


\section{Rozdělení práce}

Protože kompletní síťový simulátor pro počítače s Cisco IOS i OS Linux by přesahoval rozsah jedné bakalářské práce, byla práce na aplikaci rozdělena na tři části:
\begin{itemize}
 \item \textbf{Jádro aplikace}\\ 
Jedná se především o datové struktury virtuálního počítače, komunikaci s uživatelem po síti a načítání a ukladání souborů. Na této části jsem spolupracoval se Stanislavem Řehákem.
 \item \textbf{Linuxová část}\\
V této části je potřeba napsat parsery linuxových příkazů a  zjistit, jak se chovají počítače s linuxem v síťové komunikaci a toto chování implementovat.
 \item \textbf{Ciscová část}\\
Touto částí se tato práce nezabývá, zabývá se jí bakalářská práce Simulátor virtuální počítačové sítě Cisco Stanislava Řeháka.
\end{itemize}


\section{Struktura práce}

Tady bude popis struktry týhle práce



