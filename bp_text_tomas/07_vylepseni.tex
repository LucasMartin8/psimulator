\chapter{Možná další vylepšení}

Simulátor by se dal v mnohých ohledech ještě vylepšit. Některá vylepšení by přinesla větší uživatelský komfort, jiná by simulátor učinila věrnější skutečné síti. Zde uvádím několik námětů k vylepšení:

% \begin{enumerate}
% \item Prvním by mohlo být grafické rozhraní pro tvorbu konfiguračního souboru (struktury počítačové sítě), kde by bylo možné přidávat počítače a propojení mezi nimi.                                                                                                                                                                     
% 
% \item Pro lepší ladění problémů při konfiguraci sítě by se jistě hodil i tcpdump. Tento program funguje jako analyzátor monitorující síťový provoz na daném rozhraní.
% 
% \item Aplikace podporuje pouze jeden síťový prvek - směrovač, a tak by bylo možné program rozšířit o~další prvky např. switch (přepínač) nebo bridge. Přidání těchto prvků by znamenalo plnohodnotnou implementaci 2. linkové vrstvy ISO/OSI modelu.
% 
% \item Dalším vylepšením by mohla být možnost propojit virtuální se skutečnou sítí. 
% 
% \item Aplikace de facto nezpracovává signály (Ctrl+C a Ctrl+Z), pouze je přeposílá operačnímu systému (tuto funkci zajišťuje rlwrap). Pokud bychom se chtěli vracet do privilegovaného módu pomocí Ctrl+Z, tak by bylo nutné implementovat vlastního klienta.
% \end{enumerate}

\begin{itemize}
\item Dle mého názoru je největším nedostatkem naší aplikace komunikace simulátoru s klientem. Především by bylo dobré najít nebo naprogramovat vlastního, třeba i grafického, klienta, pomocí kterého by se uživatelé k simulátoru připojovali. Zvláště pod OS Windows je  k připojování k aplikaci pomocí emulátoru cygwin velmi nepohodlné. Další obrovskou nevýhodou stávající komunikace je nemožnost posílat signály jako \verb|Ctrl+C| nebo \verb|Ctrl+Z|.
\item Uživateli by velmi pomohlo, kdyby si topologii sítě mohl vytvořit v nějakém, nejlépe grafickém programu. Také zobrazování topologie již vytvořené sítě by bylo velmi užitečným vylepšením.
\item Simulátor nepodporuje žádné síťové prvky kromě směrovačů na síťové vrstvě ISO/OSI modelu, pořádně není implementována ani linková vrstva tohoto modelu. Na síťovém rozhraní může být zatím jen jedna IP adresa. Pro potřeby předmětu PSI nejsou tyto funkcionality důležité, ale pro jiné účely by takovéto vylepšení bylo asi nutné. V tomto ohledu jsem ale realista a myslím si, že toto vylepšení nebude asi nikdy potřeba.
\item Některé linuxové příkazy nejsou zcela věrné příkazům na skutečném linuxu. Bylo by dobré opravit například parser u příkazů \verb|ifconfig|, \verb|ping| nebo \verb|traceroute|.
\end{itemize}
