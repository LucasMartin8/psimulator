% V~této práci se podařilo navrhnout a implementovat simulátor virtuální počítačové sítě cisco. Ve spojení s~kolegovu částí je tento program schopen úspěšně odsimulovat virtuální síť složenou ze směrovačů cisco a z~počítačů založených na jádru linux. V~takové síti je možné otestovat znalosti studentů týkajících se konfigurace rozhraní, statického směrování, dynamického a statického překladu adres. Z~výsledků uživatelského testování vyplývá, že provedení takovýchto testů smysl má.
% 
% Nepřipravenost studentů na laboratorních cvičení z~předmětu Y36PSI Počítačové sítě je v~současné době problém. Obyčejní studenti zpravidla nemají přístup ke směrovačům od Cisco Systems, a tak by nasazení této aplikace jako studijní pomůcky mohlo vést ke zvýšení připravenosti těchto studentů\footnote{Za předpokladu, že studenti budou mít motivaci k~získání bodů ze cvičení.} na laboratorní cvičení.


\chapter{Závěr}

V této práci se mi společně s mým kolegou Stanislavem Řehákem povedlo navrhnout a implementovat jednoduchý síťový simulátor použitelný pro výukové účely předmětu PSI. Studenti kteří nemají žádné zkušenosti s konfigurací síťových prvků, mají možnost vyzkoušet si konfiguraci na svém vlastním počítači ještě před bodovanou laboratorní úlohou. Informace o průchodu paketu, vypisované aplikací, můžou být studentům velmi užitečné při hledání chyb v konfiguraci.  